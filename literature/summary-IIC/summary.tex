\documentclass[]{article}
\usepackage[left=3cm,right=3cm,top=1.5cm,bottom=2cm,includeheadfoot]{geometry} 
\usepackage{babel}
\usepackage{hyperref}
\usepackage{mwe}
\usepackage[markcase=noupper% Großschreibung unterdrücken
]{scrlayer-scrpage}
\usepackage{amsmath}
\usepackage{amssymb}
\usepackage{hyperref}

\KOMAoptions{
	headsepline = true
}
\ihead{Tilman Hinnerichs}
\ohead{Maximization of Mutual Information for Information Clustering}
%opening
\title{Summary on the topic of Maximization of Mutual Information for Information Clustering}
\author{Tilman Hinnerichs\\
\href{mailto:tilman@hinnerichs.com}{tilman@hinnerichs.com}}

% Build subsubsubsection
\usepackage{titlesec}
\usepackage{hyperref}

\titleclass{\subsubsubsection}{straight}[\subsection]

\newcounter{subsubsubsection}[subsubsection]
\renewcommand\thesubsubsubsection{\thesubsubsection.\arabic{subsubsubsection}}
\renewcommand\theparagraph{\thesubsubsubsection.\arabic{paragraph}} % optional; useful if paragraphs are to be numbered

\titleformat{\subsubsubsection}
{\normalfont\normalsize\bfseries}{\thesubsubsubsection}{1em}{}
\titlespacing*{\subsubsubsection}
{0pt}{3.25ex plus 1ex minus .2ex}{1.5ex plus .2ex}

\makeatletter
\renewcommand\paragraph{\@startsection{paragraph}{5}{\z@}%
	{3.25ex \@plus1ex \@minus.2ex}%
	{-1em}%
	{\normalfont\normalsize\bfseries}}
\renewcommand\subparagraph{\@startsection{subparagraph}{6}{\parindent}%
	{3.25ex \@plus1ex \@minus .2ex}%
	{-1em}%
	{\normalfont\normalsize\bfseries}}
\def\toclevel@subsubsubsection{4}
\def\toclevel@paragraph{5}
\def\toclevel@paragraph{6}
\def\l@subsubsubsection{\@dottedtocline{4}{7em}{4em}}
\def\l@paragraph{\@dottedtocline{5}{10em}{5em}}
\def\l@subparagraph{\@dottedtocline{6}{14em}{6em}}
\makeatother

\setcounter{secnumdepth}{4}
\setcounter{tocdepth}{4}

\begin{document}

\maketitle

This summary is a summary of the talk given at the February 21st 2020 at MPI MiS by Tilman Hinnerichs.
\section{Clustering}
Clustering algorithms are a group of algorithms for the exploration of similarity structures among datasets. It aims on identifying groups within the data, that are somehow similar to each other. As no additional data is used, these are classified as unsupervised learning methods. Crucially, these so called clusters do not have any label, are only based on the underlying information of each data point. In general, whether a clustering is good or bad can only be determined by an expert, or ground truth. Alternatively, quality of clusters can be measured by various dataset independent measures, e.g., average r adius of the clusters and distance between different clusters. The various algorithms differ in their interpretation of similarity. \\
\section{k-means-clustering}
k-means is one of the simplest and widely known clustering algorithms and shall thus be explained here briefly. $k$ hereby defines the number of clusters.\\
The steps are the following:
\begin{enumerate}
	\item Randomly initialize $k$ points as initial means 
	\item Assign each data point to the closest of the $k$ means
	\item For each of the sets of data points assigned to one of the $k$ means, calculate the mean among each group of assignments
\end{enumerate}
Repeat steps 2 and 3 until the means are not changing anymore or within a certain margin.

\section{Discussed papers}
When trying to find a suitable representation, we are trying to have as much information from the input within the output, i.e. loose as little information from the input as possible. Thus, the \textit{transinformation} or \textit{mutual information} (MI) has to be maximized also known as the infomax principle (Linsker, 1988). Hence, making MI the objective function for representation learning seems to be a suitable approach. However,  mutual information is generally quite difficult to compute, especially in continuous and high dimensional settings.\\
Let $\mathcal{X}, \mathcal{Y}$ be the domain and the range of a (continuous and (almost everywhere) differentiable) parametric function
$E_\psi:\mathcal{X}\rightarrow\mathcal{Y}$ with parameters $\psi$. Additionally, let $\mathcal{E}_\Psi=\{E_\psi\}_{\psi\in\Psi}$ over $\Psi$. \\
Different approaches are trying to maximize mutual information between different points of the network, e.g., between input and the output, or between different representations.

\subsection{Learning Deep Representations by mututal Inforamtion Estimation and Maximization -- Deep InfoMax (DIM)}
\subsubsection{Approach}
Find the set of parameters $\psi$, such that the mutual information, $\mathcal{I}(X;E_\psi(X))$ is maximized. Ideally the marginal distribution induced by forwarding samples from the empirical distribution for a given $\psi$ should match a prior distribution, thus encouraging the output of the encoder to have desired characteristics, e.g., independence.\\
The authors maximize the mutual information between spatially-preserved features, that have to be precomputed and their high-level representation. DIM simultaneously estimates and maximizes the mutual information.\\

The authors present various techniques to estimate the MI, eventually deciding for the Noise-Contrastive Estimation (infoNCE).\\

Deep Infomax' objective function can be written as follows:
\begin{equation}
	\arg \min_{\omega_1, \omega_2, \psi} ( \alpha\hat{\mathcal{I}}_{\omega_1,\psi}(X, E_\psi(X)) +\frac{\beta}{M^2}\sum_{i=1}^{M^2} \hat{\mathcal{I}}_{\omega_2,\psi}(X^{(i)}, E_\psi(X)) ) + \arg \min_\psi \arg \max_\phi \gamma\hat{\mathcal{D}}_\phi(\mathbb{V}|\mathbb{U}_{\phi,\mathbb{P}})
\end{equation}
Hereby the following notations are used:
\begin{itemize}
	\item[$\hat{\mathcal{I}}_{\omega,\phi}$] Approximation of mutual information determined by a discriminator with parameters $\omega$
	\item[$M^2$] number of features with preserved locality extracted from the input images
	\item[$X^{(i)}$] Feature $i$ of input image $X$
	\item[$\alpha,\beta,\gamma$] weights to each part of the objective function, chosen as hyperparameters. Choices in these hyperparameters affect the learned representations in meaningful ways.
	\item[$\hat{\mathcal{D}}$] discriminator for the minimization of KL-divergence between output distribution $\mathbb{U_{\psi,\mathbb{P}}}$ and prior distribution with certain desired properties $\mathbb{V}$.
\end{itemize}

This equation can be divided ino parts that compute the cope 
\begin{enumerate}
	\item global, and
	\item local MI maximization, and
	\item prior matching
\end{enumerate}
The global part approximates the mutual information between the whole input image and its representation, as depicted in Figure 1 of the paper. For training, the first discriminator tries to determine whether the given global feature matches the given input image, which can be either the real or a random one. The global feature hereby is the output of the neural network and thus its representation. \\
For the local information, a second discriminator tries to distinguish between real and fake local feature vectors, concerning the given global feature, hence checking whether this local feature fits the global one.\\
A third discriminator is used to minimize the Kullback-Leitner-divergence between some prior with some wanted property, that the output shall gain, and the output of the network.

\subsubsection{Results}
The authors find that incorporating knowledge about locality in the input significantly improved performance in downstream tasks. It outperforms the very limited amount of other unsupervised clustering algorithms.\\

Major drawbacks are:
\begin{itemize}
	\item Only applicable to images and
	\item Needs preprocessing creating the images features
	\item needs k-means after representation calculation for eventual clustering, and is thus prone to degeneration
	\item calculates over continuous random variables, which requires complex estimators, making an approximation necessary
	\item DeepInfomax estimator reduces to entropy (stated by IIC paper)
	\item Doesn't give a hint on best amount of classes, hence making the number of clusters a hyperparameter for k-means
\end{itemize}

\subsection{Invariant Information Clustering}
\subsubsection{Approach}

The IIC paper focuses on two issues that previous papers, such as DIM and IMSAT, weren't addressing. \\

The first issue is the general degeneracy of clustering solutions. As widely known, k-means degenerates for unfortunate initializations, thus leading two one cluster capturing the whole data. This issue is addressed through the nature of the mutual information, as it is not maximized for degenerate solutions. Second, noisy data with unknown or distractor classes such as in STL10 bother some clustering algorithms. This is addressed through the concept of auxiliary overclustering.\\

Let $x, x' \in \mathcal{X}$ be paired data sample from a joint probability distribution $P(x,x')$, e.g., two images containing the same image. We now want to maximize the mutual information between representation of those similar entities, dismissing instance specific information and keeping similarities. The goal is to make the representations of paired samples the same, which is not the same as minimizing representation distance.
\begin{equation}
	\max_\psi \mathcal{I}(E_\psi(x),E_\psi(x'))
\end{equation}

In contrast to DIM the representation space $\mathcal{Y} = \{1,\dots,C\}$ is discrete and finite, hence making the calculation of the MI precise and fast. \\ 

We denote the output $E_\psi(x)$, that can be interpreted as the distribution of a discrete random variable $z$ over $C$ classes with $P(z=c|x) = E_{\psi,c} \in [0,1]^C$.\\
Now consider a pair of such  cluster assignment variables $z, z'$ for two inputs $x,x'$. Their conditional joint distribution is given by $P(z=c, z'=c'|x,x') = E_{\psi,c}(x) \cdot E_{\psi,c'}(x)$.\\

Thus, one is able to build a $C\times C$ matrix $\mathbf{P}$, with $\mathbf{P}_{cc'} = P(z=c|z'=c')$, computed by
\begin{equation}
	\mathbf{P} = \frac{1}{n}\sum_{i=1}^{n}E_{\psi,c}(x_i) \cdot E_{\psi,c}(x_i')^T
\end{equation}

Why degenerate solutions are avoided: $I(z,z') = H(z) - H(z|z')$ ($H(z)$ maximized when all equally likely to happen, $H(z|z')$ is $0$ when assignments are exactly predictable from each other. This encourages deterministic one-hot predictions. Whether the smoothness of the clustering comes from the effects of that formula or from the model itself has yet to be determined.\\

As for totally unsupervised tasks no labels are available and thus no information about pairs among the input data is present, perturbation of that input is used. The objective function becomes
\begin{equation}
\max_\psi\mathcal{I}(E_\psi(x),E_\psi(gx))
\end{equation}
where $g$ denotes a random perturbation, such as rotation, skewing, scaling or flipping (and other).\\

Furthermore, IIC is capable of auxiliary overclustering, adding additional output heads, that are trained with the whole dataset. These capture unrelevant or distractor classes, or noise within the data.\\

\subsubsection{Results}
Benefits:
\begin{itemize}
	\item Only reliant on the information itself
	\item As range is discrete, exact computation of MI is possible and fast
\end{itemize}

Content:
\begin{itemize}
	\item semantic clustering vs. representation learning
	\item theoretical foundations of IIC in statistical learning
\end{itemize}
\begin{itemize}
	\item Set state of the art performance on various dataset
\end{itemize}


\section{Wasserstein distance}

\section{Sinkhole}

\section{Image deformation}

\end{document}